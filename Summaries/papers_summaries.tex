\documentclass[reprint,amsmath,amssymb,aps]{revtex4-2}

\usepackage{graphicx}          % figures
\usepackage{tikz}              % TikZ drawings
\usepackage{pgfplots}          % plots with pgfplots
\pgfplotsset{compat=1.18}      % current pgfplots version

\usepackage{booktabs}          % nice tables (\toprule, \midrule, etc.)
\usepackage{hyperref}          % hyperlinks + cross-references
\usepackage{caption}           % better caption control
\usepackage{algorithm}         % floating algorithm environment
\usepackage{algpseudocode}     % pseudocode

\usepackage{microtype}         % improves typography (highly recommended with REVTeX)

\begin{document}
\raggedbottom

\title{Summaries of Papers for FELIX Internship 2026}

\author{Stan Daniels}
\affiliation{Radboud University Nijmegen\\FELIX Laboratory}

\date{\today}

\maketitle

%=============================================================
% CORE FELIX / KIRILYUK GROUP PAPERS
%=============================================================

\section{Kirilyuk 2010 Ultrafast optical manipulation of magnetic order}
\cite{Kirilyuk2010}

\section{Gidding 2023 Dynamic self-organisation and pattern formation by magnon-polarons}
In this article \cite{Gidding2023},
it is shown that some peculiar patterns arise when a sample is hit by a pump pulse to change its magnetic order.\\\\
Realy precise switching of magnetic order(spins) is important for low energy cost data storage.
The abstract states that it is a known fact that when a sample is hit by a ultra short pump pulse the resulting magenitization is chaotic.
This would be due to internal instabilities in the sample.
However it turns out that the bahavior of the magnetic reagion is not necceraly chaotic at all,
some patterns will arise in these cases.
It is also well understood that a spatially-localised perturbation creates propagating waves with wave vectors determined by the profile of the excitation.\\\\
I now wonder what could be the cause of these peculier paterns and how they might be used in the future.
I also still don't know what magnon-polarons are.

\section{Kwaaitaal 2024 Epsilon-near-zero regime enables permanent ultrafast all-optical reversal of ferroelectric polarization}
In this article \cite{Kwaaitaal2024},
it is shown that ultrafast excitation under epsilon-near-zero (ENZ) conditions can permanently reverse ferroelectric polarization between stable states.\\\\
ENZ materials have a dielectric constant $\epsilon \approx 0$ whch enhances light-matter interactions.
This article shows that in ENZ conditions it is possible to achieve permanent all optical switching of an order parameter.
This means that only light is used to achieve a switching of an order parameter,
the order parameter in this article is the ferroelectric polarization.
After switching from one order to the other the polarization remains stable and thus permanent.

\section{Davies 2024 Phononic switching of magnetization by the ultrafast Barnett effect}
In this article \cite{Davies2024},
It is shown that spontanious magnetization can be achieved using the ultrafast Bernett effect.\\\\
This is done through the resonant excitation of circularly polarized optical phonons in a paramagnetic substrate.
The Barnett effect describes how an inertial body with zero net magnetic momentum can aquire magnetization when mechanicaly spinning.
When the substrate is circularly polarized it generates a magnetic field that can permanently and selectively change the magnetization of the upper layer.
This effect only happens when the laser frequincy is in resonance with the phononmodiof the substrate.

\section{Stupakiewicz 2021 Ultrafast phononic switching of magnetization}
\cite{Stupakiewicz2021}

%=============================================================
% CLASSIC PAPERS IN ULTRAFAST MAGNETISM
%=============================================================

\section{Beaurepaire 1996 Ultrafast Spin Dynamics in Ferromagnetic Nickel}
\cite{Beaurepaire1996}

\section{Kimel 2005 Non-thermal optical control of magnetization in ferromagnetic semiconductors}
\cite{Kimel2005}

\section{Bigot 2009 Coherent ultrafast magnetism induced by femtosecond laser pulses}
\cite{Bigot2009}

%=============================================================
% PHONONICS & NONLINEAR LATTICE CONTROL
%=============================================================

\section{Forst 2011 Nonlinear phononics as an ultrafast route to lattice control}
\cite{Forst2011}

\section{Nova 2017 An effective magnetic field from optically driven lattice vibrations}
\cite{Nova2017}

%=============================================================
% BACKGROUND BOOKS
%=============================================================

\section{Magnetism: From Fundamentals to Nanoscale Dynamics}
\cite{StohrSiegmann}

\section{Introduction to Solid State Physics}
\cite{Kittel}

\section{Nonlinear Optics}
\cite{Boyd}

\section{Magnetic Domains: The Analysis of Magnetic Microstructures}
\cite{HubertSchafer}

\section{Quantum Optics}
\cite{ScullyZubairy}

%=============================================================
% ADDITIONAL WORKS FROM ANDREI KIRILYUK & THE FELIX GROUP (2017–2025)
%=============================================================

\section{Kimel 2019 Nonthermal optical control of magnetism and ultrafast spintronics}
\cite{Kimel2019_inverseFaraday}

\section{Mishra 2021 Plasmon-enhanced ultrafast demagnetization in magnetophotonic nanostructures}
\cite{Mishra2021_surfaceplasmon}

\section{Savoini 2018 Tracing the ultrafast magnetic response with resonant X-ray diffraction}
\cite{Savoini2018_multimode}

\section{Schubert 2017 Sub-cycle control of terahertz high-harmonic generation by dynamical Bloch oscillations}
\cite{Schubert2017_opticalanisotropy}

\section{Kalashnikova 2018 Ultrafast lattice control of magnetic anisotropy in orthoferrites}
\cite{Kalashnikova2018_garnet}

\section{Henighan 2016 Generation mechanism of THz-frequency coherent acoustic phonons in Fe by ultrafast optical excitation}
\cite{Henighan2016_phononswitch}

\section{Mishra 2020 Ultrafast demagnetization and spin transport in Co/Pt multilayers}
\cite{Mishra2020_multilayers}

\section{Ciuciulkaite 2021 Coherent control of optical phonons in iron garnet films}
\cite{CI2021_iron_garnet_modes}

\section{Mishra 2019 Dynamic regimes of multi-shot all-optical switching in ferrimagnetic alloys}
\cite{Mishra2019_AOS_multishot}

\section{Savoini 2020 Spin-lattice relaxation at ultrafast timescales investigated via resonant X-ray scattering}
\cite{Savoini2020_spinlattice} 

\bibliographystyle{apsrev4-2}
\bibliography{bib/bibliography}

\end{document}