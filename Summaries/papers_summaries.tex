\documentclass[reprint,amsmath,amssymb,aps]{revtex4-2}

\usepackage{graphicx}          % figures
\usepackage{tikz}              % TikZ drawings
\usepackage{pgfplots}          % plots with pgfplots
\pgfplotsset{compat=1.18}      % current pgfplots version

\usepackage{booktabs}          % nice tables (\toprule, \midrule, etc.)
\usepackage{hyperref}          % hyperlinks + cross-references
\usepackage{caption}           % better caption control
\usepackage{algorithm}         % floating algorithm environment
\usepackage{algpseudocode}     % pseudocode

\usepackage{microtype}         % improves typography (highly recommended with REVTeX)

\begin{document}
\raggedbottom

\title{Summaries of Papers for FELIX Internship 2026}

\author{Stan Daniels}
\affiliation{Radboud University Nijmegen\\FELIX Laboratory}

\date{\today}

\maketitle

%=============================================================
% CORE FELIX / KIRILYUK GROUP PAPERS
%=============================================================

\section{Kirilyuk 2010 Ultrafast optical manipulation of magnetic order}
This paper \cite{Kirilyuk2010} is a revieuw paper,
that summarizes progress in the field of laser manipulation of magnetic order.
This field is of great importance as memory writing speed lacks behind other techonologies,
thus creating a technological bottleneck. Therefore ultrafast femtosecond magnetic memory is important research.\\\\
\subsection{Introduction}
In magnetic storage devices,
the tradiginal zero's and one's are represented by pointing the magnetization vector of certain domains either up or down.
Traditionally such switching of the magnetic domains has been done by applying an external magnetic field,
but there appears to be a speed limit to this as explored by experiments in the Stanford Linear Accelerator\cite{Tudosa2004_UltimateSpeed}.
Thus as the traditional methods were to slow, new methods got invented by using laser pulses.
One of the first was a demagnetization by a $60fs$ laserpulse\cite{Beaurepaire1996}.\\\\
The effect a pump laser pulse has on magnetization can be catogarized in different types.
The first one being thermal effects, this occurs because the mdium hit by the pump laser absorbs the photons and thus heats up.
The change in the magnetization corresponds to that of spin temperature.
There is no direct photon-spin coupling, so first the electron and phonon systems gets energized,
and then this energy gets transferred to the spin system.
In itinerant ferromagnets this can occur in $50fs$, while in dielectrics this can take up to nanoseconds.
The lifetime of such thermal effects is given by external parameters,
such as thermal conductivity of a substrate as well as the geometry of the sample.\\
The second effect Nonthermal (photomagnetic) effects involving the absorption of pump photons.
In this case the photons are absorbed via certain electronic states that have a direct influence on magnetic parameters.
These changes are instatanious, so they occur as fast as the rise time of the pump lasser.
The change in parameters cause a change in magnetic moments.\\
And finally, there are nonthermal optomagnetic effects that do not require the absorption of pump photons,
but are based on an optically coherent stimulated Raman scattering mechanism.
Raman scattering is an inelastic, coherent light-scattering process that can transfer energy and angular momentum between photons and internal excitations (such as spins),
enabling ultrafast, nonthermal control of magnetization without photon absorption.
\subsection{Theoretical considerations}
.

\section{Gidding 2023 Dynamic self-organisation and pattern formation by magnon-polarons}
In this article \cite{Gidding2023},
it is shown that some peculiar patterns arise when a sample is hit by a pump pulse to change its magnetic order.\\\\
Realy precise switching of magnetic order(spins) is important for low energy cost data storage.
The abstract states that it is a known fact that when a sample is hit by a ultra short pump pulse the resulting magenitization is chaotic.
This would be due to internal instabilities in the sample.
However it turns out that the bahavior of the magnetic reagion is not necceraly chaotic at all,
some patterns will arise in these cases.
It is also well understood that a spatially-localised perturbation creates propagating waves with wave vectors determined by the profile of the excitation.\\\\
I now wonder what could be the cause of these peculier paterns and how they might be used in the future.
I also still don't know what magnon-polarons are.

\section{Kwaaitaal 2024 Epsilon-near-zero regime enables permanent ultrafast all-optical reversal of ferroelectric polarization}
In this article \cite{Kwaaitaal2024},
it is shown that ultrafast excitation under epsilon-near-zero (ENZ) conditions can permanently reverse ferroelectric polarization between stable states.\\\\
ENZ materials have a dielectric constant $\epsilon \approx 0$ whch enhances light-matter interactions.
This article shows that in ENZ conditions it is possible to achieve permanent all optical switching of an order parameter.
This means that only light is used to achieve a switching of an order parameter,
the order parameter in this article is the ferroelectric polarization.
After switching from one order to the other the polarization remains stable and thus permanent.

\section{Davies 2024 Phononic switching of magnetization by the ultrafast Barnett effect}
In this article \cite{Davies2024},
it is shown that spontanious magnetization can be achieved using the ultrafast Bernett effect.\\\\
This is done through the resonant excitation of circularly polarized optical phonons in a paramagnetic substrate.
The Barnett effect describes how an inertial body with zero net magnetic momentum can aquire magnetization when mechanicaly spinning.
When the substrate is circularly polarized it generates a magnetic field that can permanently and selectively change the magnetization of the upper layer.
This effect only happens when the laser frequincy is in resonance with the phononmodiof the substrate.

\section{Stupakiewicz 2021 Ultrafast phononic switching of magnetization}
In this article \cite{Stupakiewicz2021},
it is shown that certain patterns arise when we magnetize a sample using a laser.\\\\
The patterns rotates as the polarization direction of the laser rotates.
The collective excitation modes (like magnons and phonons) define the energy range that determines all the important and intriguing thermodynamic and macroscopic properties of solids,
such as electric, magnetic or crystallographic order, and the superconducting transition temperature.
Control of the crystal structure of materials in the core aim of the field of straintonics.
For the experiment they used an accumulation of pump pulses to not damage the sample but still get results.
The multi pulse approach seems to only have the effect of a slight growth of the domains.
The best magnetic switching occurs at wavelengths of $\lambda=14\mu m$ which also shows a big phonon response.
It seems tha the LO phonon are more responsible for the magnetic switching than the TO phonons,
but i don't know what they mean, further investigation required.\\\\
In the future, ultrafast modification of the crystal field environment, and thus of magnetocrystalline anisotropy,
may become the most universal way to manipulate magnetization.
Magneto-elastic interactions are present in all materials and thus can be used everywhere, for example in antiferromagnets. 

%=============================================================
% CLASSIC PAPERS IN ULTRAFAST MAGNETISM
%=============================================================

\section{Beaurepaire 1996 Ultrafast Spin Dynamics in Ferromagnetic Nickel}
In this article \cite{Beaurepaire1996},
the relaxation processes of electrons and spins systems following the absorption of femtosecond optical pulses in ferromagnetic nickel have been studied.
They have used pump probe techniques and have shown that the magnetization drops rapidly after just a few pico-seconds.\\\\
They talk about the kerr effect, I should investigate what it is exactly.
The magneto optical kerr effect is known as MOKE.
The aim of this paper is to study both electronic and spin dynamics after excitation of a Ni film with 60 fs pulses.
The delays between pump and probe are achieved using a modified Michelson interferometer (should learn what this is).
The signals are recorded using a boxcar and a lock-in synchronous detection.
The information about the spin dynamics is contained in the time evolution of the hysteresis loops recorded for each time delay $\delta t$.
hysteresis loop basically means that the output depends on the history of the input.\\\\
This work basically showed that using optical and magnetooptical techniques cann be used to measure extremely fast events.
The experiment showed that for the first few picoseconds the dynamics of spin and electron temperatures are different.

\section{Tudosa 2004 The ultimate speed of magnetic switching in granular recording media}
\cite{Tudosa2004_UltimateSpeed}

\section{Kimel 2005 Non-thermal optical control of magnetization in ferromagnetic semiconductors}
In this article \cite{Kimel2005},
the use of the inverse Faraday effect to control spin dynamics of magnets,
via circularly polarized femtosecond laser pulses has been found.\\\\
Previous to this work the fastest way to do this was by switching magnetic domains via a rapid temperature increase.
This gave a limitation on potential applications as it was limited by thermal cooling time.
The measurements were performed in a pump and probe configuration using an amplified 200fs pulses.
The material studied was dysprosium orthoferrite $DyFeO_3$\\\\
The Faraday effect is observed as a rotation of the polarization plane of light transmitted through a magnetic medium.
\[
    \alpha_{\mathcal{F}} = \frac{\chi}{n} M * K
\]
Where $\alpha_{\mathcal{F}}$ is the specific Faraday rotation, $M$ is the magnetization, $n$ is the refractive index,
$K$ is the wave vector of light and $\chi$ is the magneto-optical susceptibility, which is a scalar value.
There is also the inverse Faraday effect, where high-intensity laser radiation induces a static magnetization.
\[
    M(0) = \frac{\chi}{16\pi}(E(\omega) \times E^*(\omega))
\]
Where $M(0)$ is the static magenitization,
$E(\omega)$ and $E^*(\omega)$ the electric field of the light wave and its complex conjugate.

\section{Bigot 2009 Coherent ultrafast magnetism induced by femtosecond laser pulses}
In this article \cite{Bigot2009},
the coupling of ultrashort laser pulses to the spin of electrons in ferromagnetic metals is investigated.
It is shown that a single 50-fs laser pulse couples efficiently to a ferromagnetic film during its own propagation.\\\\
They demonstrated that by carrying out a single-pulse magneto-optical Kerr and Faraday experiment,
that ferromagnetic states of thin films of $Ni$ and $CoPt_3$ is modified during the propagation of the laser pulse. 
For high laser excitation, the thermalization time $\tau_S$ of the spins is longer than that of the electrons $\tau_e$,
showing their different thermodynamics.\\\\
The pump-probe experiment gives rise to different processes.
The first one is the interaction of the photon field with the electronic charges and spins,
here the angular momentum of light must be modified nonlinearly during this temporal regime of less than $50fs$.
Secondly the relaxation of the electrons and spins to thermalized populations takes place.
During this the thermal relaxation of the electrons and spins leads to the demagnetization of the ferromagnet. 

%=============================================================
% PHONONICS & NONLINEAR LATTICE CONTROL
%=============================================================

\section{Forst 2011 Nonlinear phononics as an ultrafast route to lattice control}
In this article \cite{Forst2011},
an expiremtal demonstration of the ionic Raman scattering(IRS) was reported,
using femtosecond excitation and coherent detection of the lattice response.\\\\
Before this paper 2 types of coupling between electromagnetic radiation and a crystal lattice were known.
The first is the direct coupling of light to infrared-active vibrations carrying an electric dipole.
The second is indirect, involving electron-phonon coupling and occurring through excitation of the electronic system,
stimulated Raman scattering is an example of this.
IRS is a process that relies on lattice anharmonicities rather than electron-phonon interactions.
IRS opens up a new direction for the optical control of solids in their electronic ground state.\\\\
Crystal lattices respond to mid-infrared radiation with oscillatory ionic motions along the eigenvector of the resonantly excited vibration.
For IRS, the coupling of the infrared-active mode to Raman-active modes is described by the Hamiltonian:
\[
    \mathcal{H}_A = -NAQ_{IR}^2Q_{RS} 
\]
Where $N$ is the number of cells, $A$ is the anharmonic constant,
$Q_{IR}$ is the normal coordinate and $Q_{RS}$ is the coordinate of a Raman-active mode of frequency $\Omega_{RS}$.
We can make a equation of motion for the Raman mode:
\[
    \ddot{Q}_{RS} + \Omega_{RS}^2 Q_{RS} = A Q_{IR}^2
\]
So the coherent nonlinear response of the lattice results in rectification of the infrared vibrational field,
with the concomitant excitation of a lower-frequency Raman-active mode.\\\\
The experiments were performed on a $La_{0.7}Sr_{0.3MnO_3}$ with a Curie temperature of $T_C \approx 350k$.
The sample was held at a base temperature of $14K$, in its ferromagnetic phase,
and was excited using femtosecond mid-infrared pulses tuned between $9\mu m$ and $19\mu m$,
at fluences up to $2 mJcm^{-2}$.
The pulse duration was determined to be $120fs$ across the whole spectral range used here.


\section{Nova 2017 An effective magnetic field from optically driven lattice vibrations}
\cite{Nova2017}

%=============================================================
% BACKGROUND BOOKS
%=============================================================

\section{Magnetism: From Fundamentals to Nanoscale Dynamics}
\cite{StohrSiegmann}

\section{Introduction to Solid State Physics}
\cite{Kittel}

\section{Nonlinear Optics}
\cite{Boyd}

\section{Magnetic Domains: The Analysis of Magnetic Microstructures}
\cite{HubertSchafer}

\section{Quantum Optics}
\cite{ScullyZubairy}

%=============================================================
% ADDITIONAL WORKS FROM ANDREI KIRILYUK & THE FELIX GROUP (2017–2025)
%=============================================================

\section{Stanciu 2007 All-Optical Magnetic Recording with Circularly Polarized Light}
In this article \cite{Stanciu2007},
it is experimentally demonstrated that the magnetization can be reversed in a reproducible manner by a
single 40 femtosecond circularly polarized laser pulse, without any applied magnetic field.\\\\
The material studied was $GdFeCo$ with a saturation magnetization of about $4\pi M = 1000G$ at room temperature,
and Curie point of $T_C = 500K$.
To excite the material they used regeneratively amplified pulses from a Ti:sapphire laser at a wavelength of $\lambda = 800nm$,
and a repetition rate of $1 kHz$.
Each pulse had a Gaussian intensity profile, with a width at half maximum of $40 fs$.
The laser pulses were incident normal to the surface area.
The beam was focused down to a $100\mu m$ spot.
The laser had 3 different polarizations: $\sigma^+, L \text{ and } \sigma^-$.\\\\
Results were that the circularly polarized laser pulses($\sigma^+, \sigma^-$) could effectively switch magnetization,
linearly polarized light($L$) left a chaotic domain behind it.
Laser pulses of $40 fs$ can already switch magenitization.

\section{Kimel 2019 Nonthermal optical control of magnetism and ultrafast spintronics}
\cite{Kimel2019_inverseFaraday}

\section{Kalashnikova 2018 Ultrafast lattice control of magnetic anisotropy in orthoferrites}
\cite{Kalashnikova2018_garnet}

\section{Henighan 2016 Generation mechanism of THz-frequency coherent acoustic phonons in Fe by ultrafast optical excitation}
\cite{Henighan2016_phononswitch}

\section{Mishra 2020 Ultrafast demagnetization and spin transport in Co/Pt multilayers}
\cite{Mishra2020_multilayers}

\section{Savoini 2020 Spin-lattice relaxation at ultrafast timescales investigated via resonant X-ray scattering}
\cite{Savoini2020_spinlattice} 

%=============================================================
% SIMULATION-ORIENTED PAPERS: Laser-induced magnetization dynamics / micromagnetics / all-optical switching
%=============================================================

\section{Miao 2018 Micromagnetic Studies of Laser-Induced Magnetization Dynamics in FePt-C Films}
In this article \cite{Miao2018_FePtC_laser},
they have simulated laser-induced magnetization dynamics using a hybrid Monte Carlo micromagnetic method.
The results show that the magnetization dynamics includes an ultrafast demagnetization,
a slower magnetization recovery, and a long-timescale magnetization reversal or the continuing recovery,
depending on the magnitude of laser fluence and the external magnetic field.\\\\
They have only studied linearly polarized light,
but the experimental and simulated results closely match.
This shows that hybrid Monte carlo micromagnetic simulations are a good way to go about simulating these kinds of effects.
They have used ime-resolved magneto-optical Kerr effect (TR-MOKE) to derive the spin temperature profile directly.
In the TR-MOKE measurement, a pump pulse excites the sample,
and then the reflected probe beam is split into two orthogonal polarized components, denoted by signal A and signal B.
The difference between their intensities, IA -IB,
is due to the rotation of plane of polarization, which is induced by the magneto-optical Kerr effect.

\section{Raposo 2020 Micromagnetic Modeling of All Optical Switching of Ferromagnetic Thin Films: The Role of Inverse Faraday Effect and Magnetic Circular Dichroism}
In this article \cite{Raposo2020_AOS_micromagnetic},
a all optical switching (AOS) experiment has been simulated.
They investigated the Inverse Faraday Effect (IFE) and the Magnetic Circular Dichroism (MCD) as effects of the magnetic switching.
They found that the IFE produced domain wall movements the most accurately,
but for local inversion of the initial magnetic state no big difference has been found.\\\\
The simulation used a micromagnetic model based on the Landau-Lifshitz-Bloch equation coupled to the heat transport.
The temperature evolution in the system under the action of the laser pulses is described by the two Temperature Model (2TM),
with $T_e$ the electrons temperature and $T_l$ the lattice temperature.
So they simulated ultra short laser pulses hitting ferromagnetic thin films.
The power of the laser s describe by:
\begin{equation*}
    P(r, t) = P_0 \exp\left[- \frac{r^2}{d_0^2/(4\ln(2))}\right] \exp\left[-\frac{(t-t_0)^2}{\tau_L^2/(4\ln(2))}\right]
\end{equation*}
With $r=\sqrt{x^2+y^2}$: the distance from the laser spot center,
$d_0$: the Full Width at Half Maximum (FWHM) of laser spot, $\tau_L$: the FWHM of the pulse duration.
The maximum power of the laser is:
\begin{equation*}
    P_0 = \frac{F}{t_{FM}\tau_L}
\end{equation*}
where $F$ is the laser fluence and $t_{FM}$ is the thickness of the ferromagnetic sample.
Then as the laser heats up the sample, we use the 2TM with temperature dynamics given by:
\begin{align*}
C_e \frac{\partial T_e}{\partial t} &= - k_e \nabla^2 T_e - g_{el} (T_e - T_l) + \eta(\mathbf{m},\sigma) (1-R) P(\mathbf{r},t) \\
C_l \frac{\partial T_l}{\partial t} &= - k_l \nabla^2 T_l - g_{el} (T_l - T_e) - \frac{C_l}{\tau_{\text{sub}}} (T_l - T_{\text{sub}})
\end{align*}
With $C_e = \gamma_e T_e$ the electron heat capacity(which is linear in $T_e$),
$C_l$ the lattice heat capacity(which is constant above the debye temperature),
$k_i$ the thermal conductivity, $g_{el}$ the electron-lattice coupling,
$\mu(m, \sigma)$ the polarization- and magnetization-dependent absorption factor (MCD),
$R$ the reflectivity and $\tau_{sub}$ the heat transfer time to the substrate.
As the laser pulse usually heats the sample close or even over its Curie temperature $T_C$,
the Landau-Lifshitz-Gilbert equation (LLG) cannot be used to describe the magnetization dynamics.
Instead, the Landau-Lifshitz-Bloch (LLB) equation must be employed.
In this case, the dynamics of the normalized magnetization is given by:
\begin{align*}
\frac{d\vec{m}}{dt} &= 
- \gamma_0' \, \vec{m} \times \vec{H}_{\text{eff}}
- \gamma_0' \frac{\alpha_\perp}{m^2} \,
\vec{m} \times \big( \vec{m} \times (\vec{H}_{\text{eff}} + \vec{H}_{\text{th}}^{\,\perp}) \big) \nonumber\\
&\quad + \gamma_0' \frac{\alpha_\parallel}{m^2}
(\vec{m} \cdot \vec{H}_{\text{eff}})\, \vec{m}
+ \vec{H}_{\text{th}}^{\,\parallel}
\end{align*}
where the normalized magnetization is:
\begin{equation*}
    \vec{m}(\vec{r}, t) = \frac{\vec{M}(\vec{r, t})}{M_s^0}
\end{equation*}
With $M_s^0$ is the saturation magnetization at $T=0K$.
The longitudinal and transverse damping parameters in the LLB model are given by:
\begin{equation*}
    \alpha_\parallel = \frac{2 \lambda T}{3 T_C m_e^2},
    \qquad
    \alpha_\perp     = \lambda \left( 1 - \frac{T}{3 T_C} \right),
\end{equation*}
with $\lambda$ the microscopic coupling to the thermal bath, $T$ the local temperature
(from the 2TM), $T_C$ the Curie temperature and $m_e(T)$ the equilibrium magnetization.
The effective field entering the LLB equation consists of several contributions:
\begin{equation*}
    \vec{H}_{\text{eff}} =
    \vec{H}_{\text{ext}}
    + \vec{H}_{\text{ani}}
    + \vec{H}_{\text{ex}}
    + \vec{H}_{\text{dem}}
    + \vec{H}_{\text{long}},
\end{equation*}
where $\vec{H}_{\text{ext}}$ is the external field,
$\vec{H}_{\text{ani}}$ the uniaxial anisotropy field,
$\vec{H}_{\text{ex}}$ the exchange field,
$\vec{H}_{\text{dem}}$ the demagnetizing field,
and $\vec{H}_{\text{long}}$ the longitudinal relaxation field.
The longitudinal field is expressed as:
\begin{equation*}
    \vec{H}_{\text{long}} =
    \frac{1}{2 \chi_\parallel}
    \left( 1 - \frac{m^2}{m_e^2(T)} \right) \vec{m},
\end{equation*}
with $\chi_\parallel$ the longitudinal susceptibility.
The transverse dynamics is driven by the transverse susceptibility:
\begin{equation*}
    \vec{H}_{\perp} =
    \frac{1}{\chi_\perp}
    (\vec{m} \cdot \hat{e}_\perp) \, \hat{e}_\perp,
\end{equation*}
where $\chi_\perp$ is the transverse susceptibility and
$\hat{e}_\perp$ is the local transverse direction of deviation.
Finally, the model includes the thermal stochastic fields in longitudinal and transverse directions:
\begin{align*}
    \langle H_{\text{th},i}^{\,\perp}(t) \rangle &= 0, \quad
    \langle H_{\text{th},i}^{\,\perp}(t) H_{\text{th},j}^{\,\perp}(t') \rangle =
    \frac{2 \alpha_\perp k_B T}{\gamma_0' M_s^0 V} \delta_{ij} \delta(t-t'), \\
    \langle H_{\text{th}}^{\,\parallel}(t) \rangle &= 0, \quad
    \langle H_{\text{th}}^{\,\parallel}(t) H_{\text{th}}^{\,\parallel}(t') \rangle =
    \frac{2 \alpha_\parallel k_B T}{\gamma_0' M_s^0 V} \delta(t-t'),
\end{align*}
with $k_B$ the Boltzmann constant and $V$ the computational cell volume.\\\\
The optical excitation produced by the circularly polarized laser pulse introduces
two additional mechanisms into the micromagnetic dynamics:
the Inverse Faraday Effect (IFE) and the Magnetic Circular Dichroism (MCD).
The IFE produces an effective magnetic field acting along the laser propagation
direction, while the MCD modifies the local absorption depending on the
magnetization orientation.
The IFE field is written as:
\begin{align*}
\vec{B}_{\text{MO}}(\vec r, t) & = (\sigma_\pm)\,B^{\text{MO}}_0 \; f_{\text{MO}}(\vec r, t)\,\hat{u}_z, \\
\vec{H}_{\text{MO}}(\vec r, t) & = \frac{1}{\mu_0}\,\vec{B}_{\text{MO}}(\vec r, t),
\end{align*}
where $\sigma_\pm = \pm 1$ denotes the laser helicity (right- or left-handed circular polarization),
$B^{\text{MO}}_0$ the maximum optically induced magnetic flux density,
and $f_{\text{MO}}(\vec r, t)$ a spatiotemporal envelope similar to the laser profile:
\begin{equation*}
f_{\text{MO}}(\vec r, t) =
\begin{cases}
\exp\!\Big[-\dfrac{r^2}{d_0^2/(4\ln 2)}\Big] \\[3pt]
\times \exp\!\Big[-\dfrac{(t-t_0)^2}{\tau_L^2/(4\ln 2)}\Big],
& t < t_0,
\\[10pt]
\exp\!\Big[-\dfrac{r^2}{d_0^2/(4\ln 2)}\Big] \\[3pt]
\times \exp\!\Big[-\dfrac{(t-t_0)^2}{(\tau_L+\tau_d)^2/(4\ln 2)}\Big],
& t \ge t_0.
\end{cases}
\end{equation*}
Here $d_0$ is the laser spot FWHM, $\tau_L$ is the pulse duration FWHM,
and $\tau_d$ accounts for a possible delay of the optically induced field decay.
The IFE field therefore introduces a helicity-dependent torque which can move domain walls
or contribute to magnetization reversal.
The Magnetic Circular Dichroism (MCD) modifies the absorption of the laser power
depending on the relative orientation between the magnetization and the light helicity.
This is described by:
\begin{equation*}
\eta(\vec m, \sigma) =
A_{\text{LP}}
\Big[ 1 - \sigma_\pm \, m_z \, \text{MCD}(\%) \Big],
\end{equation*}
where $A_{\text{LP}}$ is the absorption coefficient for linearly polarized light
and $\text{MCD}(\%)$ is the dichroism amplitude.
The absorbed power density entering the 2TM heat equation then becomes
$\eta(\vec m,\sigma)(1-R)P(\vec r,t)$, introducing a helicity-dependent local heating.
Finally, both optical effects enter the Landau-Lifshitz-Bloch dynamics through the effective field:
\begin{equation*}
\vec{H}_{\text{eff}} =
\vec{H}_{\text{ext}}
+ \vec{H}_{\text{ani}}
+ \vec{H}_{\text{ex}}
+ \vec{H}_{\text{dem}}
+ \vec{H}_{\text{long}}
+ \vec{H}_{\text{MO}},
\end{equation*}
thus ensuring that the laser-material interaction is fully included in both the thermal
and magnetic evolution of the system.

\section{Aviles-Felix 2021 All-optical spin switching probability in [Tb/Co] multilayers}
\cite{Aviles-Felix2021_TbCo_AOS}

\section{Wei 20216 Micromagnetics at Finite Temperature}
This article \cite{Wei20216},
shows the development of hybrid Monte carlo (HMC) micromagnetic simulations.
The HMC micromagnetics is a self-consistent method for the magnetic studies at finite temperature.

\section{Kazantseva 2008 Slow recovery of the magnetisation after a sub-picosecond heat pulse}
\cite{Kazantseva_2008}

\section{Atxitia 2009 Ultrafast Spin Dynamics: The Effect of Colored Noise}
\cite{Atxitia2009}

\bibliographystyle{apsrev4-2}
\bibliography{bib/bibliography}

\end{document}