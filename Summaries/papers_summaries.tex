\documentclass[reprint,amsmath,amssymb,aps]{revtex4-2}

\usepackage{graphicx}          % figures
\usepackage{tikz}              % TikZ drawings
\usepackage{pgfplots}          % plots with pgfplots
\pgfplotsset{compat=1.18}      % current pgfplots version

\usepackage{booktabs}          % nice tables (\toprule, \midrule, etc.)
\usepackage{hyperref}          % hyperlinks + cross-references
\usepackage{caption}           % better caption control
\usepackage{algorithm}         % floating algorithm environment
\usepackage{algpseudocode}     % pseudocode

\usepackage{microtype}         % improves typography (highly recommended with REVTeX)

\begin{document}
\raggedbottom

\title{Summaries of Papers for FELIX Internship 2026}

\author{Stan Daniels}
\affiliation{Radboud University Nijmegen\\FELIX Laboratory}

\date{\today}

\maketitle

\section{Kirilyuk 2010 Ultrafast optical manipulation of magnetic order}
\cite{Kirilyuk2010}

\section{Gidding 2023 Dynamic self-organisation and pattern formation by magnon-polarons}
In this article \cite{Gidding2023},
it is shown that some peculiar patterns arise when a sample is hit by a pump pulse to change its magnetic order.\\\\
Realy precise switching of magnetic order(spins) is important for low energy cost data storage.
The abstract states that it is a known fact that when a sample is hit by a ultra short pump pulse the resulting magenitization is chaotic.
This would be due to internal instabilities in the sample.
However it turns out that the bahavior of the magnetic reagion is not necceraly chaotic at all,
some patterns will arise in these cases.
It is also well understood that a spatially-localised perturbation creates propagating waves with wave vectors determined by the profile of the excitation.\\\\
I now wonder what could be the cause of these peculier paterns and how they might be used in the future.
I also still don't know what magnon-polarons are.

\section{Kwaaitaal 2024 Epsilon-near-zero regime enables permanent ultrafast all-optical reversal of ferroelectric polarization}
In this article \cite{Kwaaitaal2024},
it is shown that ultrafast excitation under epsilon-near-zero (ENZ) conditions can permanently reverse ferroelectric polarization between stable states.\\\\
ENZ materials have a dielectric constant $\epsilon \approx 0$ whch enhances light-matter interactions.
This article shows that in ENZ conditions it is possible to achieve permanent all optical switching of an order parameter.
This means that only light is used to achieve a switching of an order parameter,
the order parameter in this article is the ferroelectric polarization.
After switching from one order to the other the polarization remains stable and thus permanent.

\section{Davies 2024 Phononic switching of magnetization by the ultrafast Barnett effect}
\cite{Davies2024}

\bibliographystyle{apsrev4-2}
\bibliography{bib/bibliography}

\end{document}